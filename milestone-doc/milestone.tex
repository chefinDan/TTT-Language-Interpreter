\documentclass{article}
\usepackage{listings}
\lstset{escapeinside={(*@}{@*)}, frame=single}

\begin{document}

\begin{flushleft}
CS 381 Project Milestone

Michael Andrews

Daniel Green

Nicholas Matsumoto

Quan Nguyen
\end{flushleft}

\section*{Introduction}

Our language, current titled ``Untitle Group Project'' and pronounced ``French'', is a pass-by-value imperative language.

The basic logical unit of our language is the Expression: we do not distinguish between top-level definitions, statements and expressions.  Rather, everything is composed of Expressions that can be composed arbitrarily, and which will always evaluate to some end result.  Expressions may also, naturally, alter state.  We approximate top level definitions by allowing the user to bind a function literal to a variable name at any time.  (Via, of course, an Expression.)

Most, but not all, Expressions take additional Expressions as arguments: for example, Add takes two additional Expressions.  When an Expression has Expression arguments, the arguments are evaluated (recursively if necessary) until some final value is obtained.  Evaluation is always performed left-to-right, and changes to the program state are passed along.

To give an idea of what this means, there is an expression which simply evaluates to a literal value, \texttt{Val} and another expression which simply dereferences a variable name and evaluates to the literal (if any) bound to it, \texttt{Var}.  It is perfectly valid syntax to simply use one of these as a statement unto itself, and indeed, that is how functions (and programs, which are functions themselves) return a value: the function itself evaluates to the value evaluated to by its final expression.

Consider the following pseudocode, in which \texttt{x} and \texttt{y} are initially undefined:

\begin{lstlisting}
y = (x = 3) + x++
\end{lstlisting}

In our implemented semantics, this would be written as:

\begin{lstlisting}
Assign "y" (Add (Assign "x" (Var (I 3)) (increment "x"))
\end{lstlisting}

In Untitled Group Project, this is perfectly valid.  Assign both binds a literal to a name and evaluates to the bound value; increment (which is syntactic sugar) both modifies the bound value associated with the name and evaluates to the result.  The overall expression would therefore evaluate to 7, and afterwards the name \texttt{y} would bound to the value 7, while \texttt{x} would be bound to 4.

By taking this approach, nearly everything can be composed arbitrarily.  Note that this flexibility does come at a cost, though: because nearly everything in the grammar is composed of the same Haskell type, it is extremely easy to pass an invalid argument to almost any expression: Add expects, ultimately, to get either two integers or two strings: nothing stops the user from writing a program that attempts to add a function to an error.  This necessitates robust error reporting; we used the Debug.trace library to allow programs to emit descriptive errors at runtime.

There is also the question of scope.  `State' in our language is a collection (a hashmap, in fact) of names bound to literal values.  Since functions are first-class values in our language, the state is the repository of all bound functions.  A function has in its definition a list of names (parameters;) when a function is called, a list of expressions is provided (arguments,) and these are bound pairwise to the parameter names in the function definition to form a new state, which is passed into the function call.  Variables which are themselves functions are also passed in, but variables of any other type are stripped out.  When a function returns, any functions which were bound to variables inside are passed back to the call site's state, but all other variables are discarded.  In this way, function variables are global, but all other kinds of variable are local.

We feel that at this point, the language is largely operational, and our aim is to improve usability by adding more features as sugar and as library functions.

We have tried to implement as core features only things which we feel are needed to allow the language to be sufficiently expressive, and which we have been unable to find a way to implement as sugar on existing features.  (While loops are the exception.)

\section*{Design}

\subsection*{Basic Data Types}
Our basic data types are Integers, Strings, Functions, Lists, and Booleans.  Booleans are implemented as syntactic sugar, but all others are Core features.  Our language also has a concept of Nil and Error values, but these exist only as evaluated values; they do not exist as literals.

\subsection*{Conditionals}
If-Else-Then is implemented as a Core feature.

\subsection*{Loops}
Both recursion and While loops are available, and are implemented as core features.  We plan to add for-loops as sugar.  While loops represent our only major break from orthogonality; while we recognize that they're not strictly necessary given that we have recursion, we want to have them.

\subsection*{Variables}
Mutable variables exist.  Function variables are global, all other variables are local.  Any basic data type which can be expressed as a literal (Integer, String, List, Function) can be bound to a name.  This is core.

\subsection*{Procedures/Functions with Arguments}
Functions exist, can be defined within a program, and can take arguments.  A function's scope consists initially of whatever arguments it was given and any function variables bound in the calling scope.  This is core, exept that 'defining' a function is syntactic sugar overlaid on the core variable assignment expression.

\subsection*{Strings (1 pt)}
Strings are a basic data type.  Operations on them are limited, but they can be concatenated, multiplied by an integer (which self-concatenates the requested number of times) or divided by an integer (which truncates the string in the proportion requested.)  This is core.

\subsection*{Lists (2 pts)}
Lists are available.  They can be indexed into, prepended to, appended to, and arbitrary indices overwritten.  It is possible to loop over an index, and we intend to add a map function as a library tool.

List are, of course, intrinsically invertible; anything you put into a list you can extract again at will.

Lists, and their basic operations, are implemented as core.

\subsection*{First-Class Functions}
Functions are a basic data type; they can be passed into functions, stored in lists, and bound to variables.  Trying to do arithmetic on a function will generate an error, however.

\subsection*{Additional Note}
Our language has a number of features not listed here, like boolean operations for And, Or, and Not, equality evaluation, etc.  We also plan to continue implementing additional features as sugar and library features going forward, so the above is not a comprehensive feature list, just what the milestone guidelines ask us to list.

\section*{Safety}

Due to the heavily composable nature of Untitled Group Project, there are a great many errors that can occur.  It's possible to write a valid (Haskell) program in our language in which every operation finds itself with the wrong operands.  We handle this in two ways.  Firstly, our semantic domain for evaluating expressions (and programs) allows for the possibility of returning an error.  (More on this in the next section.)  Secondly, any time an expression evaluates to an error, we take advantage of Haskell's debug tracing to print an appropriate message to the console.  Because the program will keep running, a lot of error messages can pile up, but of course, the first one emitted is the first place to look to begin fixing them.

\section*{Implementation}

Our language has two semantic domains, one for Expressions and one for Programs.  The semantic domain for Expressions is:

\begin{lstlisting}
Context -> Expression -> (Context, Result)
\end{lstlisting}

Context, in this context (ha!) is a strict hashmap that contains the state; the collection of Name/Value bindings known to the current Expression.  (Due to scoping, this may not be everything known to every part of the program.)

Result has the following definition:

\begin{lstlisting}
data Result = Valid Value | Error | Nil
\end{lstlisting}

This domain was chosen because our program has a mutable state; each individual expression must necessarily receive a copy of that state, and return a (possibly modified) state for the next expression to use.

The semantic domain for programs is:

\begin{lstlisting}
Context -> Value -> Result
\end{lstlisting}

This is akin to the domain for statement evaluation, except that we cannot take an expression (because no program is yet running to evaluate it) so we must take a raw function literal. The second argument (the Value) is the function literal to be run; if that seems odd, recall that functions are a basic data type in our language. (Attempting to run a non-function literal, is, of course, an error.)  A context is passed in, containing the library, or whatever other definitions the user wishes to pass in.  This forms the initial state for the program. Only a value is returned, because once the program has finished, the state no longer has meaning.  We also decided not to allow arguments to be passed in to programs themselves; we may revisit that, however.

\end{document}